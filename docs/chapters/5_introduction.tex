\section*{Introduction}
Human computer interaction has been evolving over the past decades, with evolving human needs. With developments in technologies, the physical 
contact between the computer and the human is decreasing rapidly. Advanced systems which work on speech and gesture control still requires a minimal
effort from the user to interact with the machine. Though these effort seem to be mere, it is a challenging task for humans with disabilities. 
Systems which work on facial gestures bridge this gap to an extent but it does not completely what is the actual intent of the person. Brain Computer
interface paves way to encode the persons intent and thoughts without the need for any physical effort. It provides enourmous capabilities for 
physicaly challenged people to express themselves just by their thoughts. 

Autonomous vehicles are the future of mobility, several companies around the world invest and research on new technologies to solve new challenges 
that appears in developing level 5 autonomy. The level of human interaction with the vehicle has been decreasing with increasing safety. However
including human in the loop is necessary at certain times to avoid any undesirable events. Level  5 autonomous vehicles is still a long way to go, 
but by bringing in a minimal interaction of the driver with the system, safety can be ensured. One of the ways of acheiving it is interfacing the
thought and decission process of the driver to the autonomous vehicle, a process commonly referrred to as Brain-Computer interface. 

Once shown in science fiction novels and movies, Brain-Computer interface has become widely researched and developed in the academic institutions 
and at industries in the past decades. It has been applied and tested on mammals for a wide range of applications. However it has its own challenges
and limitations. With evolving technologies, new innovations and discoveries are made to understand and decode the brain waves better. The analysis
of the brain signals have seen a shift in the paradigm with the introduction of machine learning techniques. This work focuses on brain signal analysis and decoding 
using reliable and conventional signal processing techniques as well as state-of-the-art deep learning techniques to extract the features and classify
it. Several open source datasets are used to setup the basic brain signal processing pipeline and later tuned to work with the data obtained form
OpenBCI headgear with 16 electrodes. The goal of this work is to navigate a car in the CARLA simulator using brain signals obtained from the OpenBCI
headgear in real-time.

Modern BCI design involves understanding brain dynamics, recognizing the patterns in the brain waves and derive the statistics of the data obtained
optimize it and use them as features to predict or classify the task. However it is very challenging to create a BCI system that can work on any person
as the brain signals are task specific and brain signal signatures are very unique to a person. The cortex fldings and the relevant functional maps are
differnet across individuals. Even for the very same person, the brain synamics are non-stationary at all time scales adding to it, it is almost impossible 
to place the electrodes exactly on the same location for every recording sessions. The signal-to-noise ratio in a brain signal is very poor, that makes it
difficult to obtain required information from brain signal. It is harder to spatially measure the data from one region as large collection of neurons
are involved in many different activity, not just one. These challenges can vary a lot depending on the methods used to record and analyse the brain signals.


\section{Brain Signal Acquisition}
Neurons are the basic computational unit of the nervous system. The electric signal recorder as brain wave is the result of electrochemical interations
that occur at the outer membrane of neurons. In case of a event, the raises and falls of potential at the membrane of the neuron is called as action potential.
The electrical activity at the neurons enable us to record and decode the brain activities. There are many ways to record the electrical activity and 
are broadly classified into- invasive and non-invasive recording.

    \subsection{Invasive Approaches}

    \subsection{NonInvasive Approaches}

\section{EEG Paradigms}

    \subsection{Event Related Potential(ERP)}

    \subsection{Error Related Potential(ErRP)}

    \subsection{Motor Imagery(MI) or Oscillatory Potential}


\section*{Summary} 