\section*{Introduction}
The data after preprocessing is clean from noise and other artifcats, now the feature extracted 
Brain signals are transient as well as diffuse spike and action potential. The include small oscillaions of 
semi periodic activations. Therefore the local information exists in time or frequency domain or both.
It is rhythemic and hence could be sinusoidal or non-sinusoidal. 
The brain signals could be time locked i.e. the activity in the brain is time synchronized with the stimulus or the temporal dynamics is the same across 
every trials at the same instance. It could be phase locked as well i.e. the phase angle time series is the same or similar on every trial.



\section{Feature Extraction}


\subsection{Time Domain Analysis}
Time domain analysis is useful in extracting temporal features in the preprocessed measurement. Some of the commonly
used time domain analysis methods include signal averaging, Hjorth parameters, Fractal dimensions, auto regressive models
Bayesian filtering, Kalman filtering, particle filtering, zero crossing, template matching, power and window detection.

Time domain analysis leads to loss of information on averaging several trials in case of non-phase locked brain signals, but 
frequency analysis doesn't lose information on averaging trials.

\subsection{Frequency Domain Analysis}
Transforming the preprocessed data into frequency domain provides more useful information on the brain signal required for analysis.
Imagined movements leads to oscillations in premotor and sensorimotor areas which is observed as amplitude changes in the $\mu$, $\beta$ and $\gamma$ region.
These features can be used to create feature vectors. Some of the common frequency domain analysis are fourier transform, short-time fourier transform 
and welch transform. These methods provide information on the spectral contents of the signal but do not proovide information on 
temporal information on the event. 

\subsubsection{Fourier Transform (FT)}
Fourier transform decomposes signal into a weighted sum of sine and cosine waves of different frequencies. Fourier transforms work on the assumptions that the
signal is infinite, periodic and stationary. Brain signals do not hold to these assumptions. These shortcomings 
can be over come by using short window periods for analysis, termed as Short Time Frequency Transform (STFT). Another commonly used method for 
frequency domain analysis is welch transform that provides smooth spectral decomposition.

Transforming time domain signals to frequency domain provides amplitude and phase information. Phase content is not used in brain signal analysis in most cases
as the signal could be non-phase locked. The amplitude spectrum is generally used in feature extraction. Power spectrum is obtained by 
square of amplitude in different frequency components during course of the task. 

\begin{equation} \label{eq:pwr}
    P(n) = A(n)^2
\end{equation}

\subsection{Time-Frequency Analysis (TFA)}
The method discussed above do not provide temporal information on occurence of the event. Time-Frequency analysis on the signal provides temporal-spectral information
on the signal. One of the most commonly used TFA method is wavelet transform

\subsubsection{Wavelet Transform (WT)}
Wavelet transform uses finite basis functions called wavelets that are scaled and translated copies of finite length waveform called mother wavelet. Wavelets divide the signals
of interest into different frequency components, each component can be studied at resultion matched for its scale. A large scale component produces coarse resolution and small 
scale component produces fine resolution. The wavelet transform provides lower frequency resolution and higher temporal resolution at higher frequencies whereas 
higher frequency resolution and lower temporal resolution at lower frequencies.

Wavelet analysis is performed by convolution of wavelets of different scale and resolution with the signal of interest. The result of convolution  contains components of the signal
that are in the same frequency of the wavelet used. Scalogram provides graphical information on the time and frequency of the signal. The transform could be either 
discrete or continuous depending upon the scale and translation parameters. Contininous Wavelet Transform (CWT) provides more resolution than the Discrete Wavelet Transform (DWT).

A generic CWT function can be written as \ref{cwt}

\begin{equation} \label{eq:cwt}
    \psi_{a,b}(t) = \frac{1}{\sqrt{a}} \psi( \frac{t - b}{a}) 
\end{equation}

where $t$ denotes time, $b$ is the shift parameter that slides the waveform along the time axis and $a$ is the scale factor that is used to scale the frequency of the wavelet.

Generally for a given wavelet function $\psi(\frac{t}{a})$ when $a > 1$ the mother wavelet dilates and it helps in capturing low frequency component and when $0 < a < 1$ it 
compresses the mother wavelet, capturing the high frequency componenets in the signal of interest. Thus $a \propto \frac{1}{f}$.

Wavelets have bandpass characteristics and its equivaent frequency can be determined by 

\begin{equation} \label{eq:wt_cf}
    F_{eq} = \frac{C_f}{a\delta_t}
\end{equation}

where $F_{eq}$ is the equivalent frequency, $C_f$ is the centre frequency, $a$ is the scaling factor and $\delta_t$ is the sampling intreval.

A generic DWT funtion can be written as \ref{dwt}

\begin{equation} \label{eq:dwt}
    \psi_{a,b}(t) = \frac{1}{2^{\frac{j}{v}}} \psi( \frac{n - m}{2^{\frac{j}{v}}}) 
\end{equation}

where $j$ are the available scales, $v  (>1)$ denotes voices per octave (lograthamic unit of ratio between two frequencies) and $m$ is the translational parameter.

% Table about wavelet scale


Various types of wavelets are used in research and one of the commonly used wavelet is the Morlet Wavelet. It the product of complex valued sine wave and gaussian.
In frequency domain,  The amplitude spectrum of the complex morlet wavelet is not symmetrical and the power results are independent of the phase relation between 
the wavelet and signal.
The Complex Morlet Wavelet can be written as \ref{Mrlt1}

\begin{equation} \label{eq:Mrlt1}
    e^{i2\pi ft -0.5(\frac{t}{\sigma})^2}
\end{equation}

where $\sigma = \frac{n}{2\pi f}$, $n$ is the number of cycles, $f $is the frequency of the complex sine wave used.

The equation \ref*{Mrlt} can also be written as  \ref*{Mrlt2}

\begin{equation} \label{eq:Mrlt2}
    e^{i2\pi ft} e^{-4\ln(2)\frac{t^2}{h^2}}
\end{equation}

where $h$ is the full width half maximum (FWHM) parameter. With higher $n$, better temporal resolution and with lower $n$ better spectral resolution is achieved. Most algorithms
work by starting with low number of cycles and gradually increasing to higher number of cycles.

However the results of the wavelet convolution are easily interpretable only for stationary signals within the FWHM of the wavelet. With the right design of
the wavelet this can be achieved.

\subsection{Wavelet Scattering Transform}

\subsection{Spatial Analysis}
The spaial filters invert the measurement to the original source
Spatial Analysis helps in mapping the source signals to the brin topography

\subsubsection{Common Spatial Patterns (CSP)}
CSP is a supervised feature extraction method for classification algorithms which works only for motor-imagery information in EEG . It learns to optimally
discriminate band power features. Variance of the band pass signal is proportional to the band power of the frequency band. CSP leverages this by finding patterns
where the  variance of filtered data from one class is maximized while variance of filtered data from other class is minimized. The resulting feature
vector enhance discriminability between different classes. However it works on the assumptions that frequency band and time window of the measurements are known, source
activity constellation differs between two classes and the band passed signal is jointly gaussian within the time window. 

Consider measurement data set $\{\mathbb{X}^{i}_{c}\}^{k}_{i=1}$ for trial $i$ belonging to class $c \in\{1,2\}$. Each $\{\mathbb{X}^{i}_{c}\}$ is a $N * T$ matrix, where
N is the number of channels and T is the number of samples per channel. The goal of CSP is to find $W$ given by $N * M$, consisting M spatial filters i.e. each column is a spatial
filter , that transforms the measured signals \ref*{trans_csp}.

\begin{equation} \label{eq:trans_csp}
    x_{CSP}(t) = W^{T}x(t)
\end{equation}

where $x(t)$ is vector of input signals at time $t$ from all channels. CSP can formally be defined for a two class problem as \ref{def_csp}.

\begin{equation} \label{eq:def_csp}
    J(w) = \frac{wX_{1}X_{1}^Tw^T}{wX_{2}X_{2}^Tw^T} \\
         = \frac{wC_{1}w^T}{wC_{2}w^T}
\end{equation}

where $C_{c}$ is the spatial covariance matrix for class $c$, w is the spatial filter to optimize and $X_{c}$ is the multichannel EEG signal from class $c$.

The problem could be solved to either minimize $J(w)$ where variance in $X_{1}$ is minimized and maximized in $X_{2}$ or maximize $J(w)$ where variance in $X_{1}$ is 
maximized and minimized in $X_{2}$.

This can be computed by Generaized Eigen Value Decomposition (GEVD) of $C_{1}$ and $C_{2}$. The Eigen vectors corresponding to the largest eigen value will maximize $J(w)$.
and eigen vectors corresponding to the smallest eigen value will minimize $J(w)$. Typically 3 CSP filter pairs i.e. minimization and maximization are used which results in
6 feature vectors that forms the filter matrix $w$.

Once filter is obtained, the prediction function using CSP can be defines as \ref*{feat_csp}

\begin{equation} \label{eq:feat_csp}
    y = sign(\theta log(var(wX)) +b) = sign(\theta log(wCw^T) + b)
\end{equation}
The application of log on the spatially filtered data makes it a gaussian distribution. The classifier applied can be either linear or nonlinear. After applying CSP 
the data is rotated and placed orthogonal.

CSP is simple, computationally efficient with high classification performance. But it is not robust to noise and nonstationaries, prone to overfitting and requires
many training examples.

Several variants of CSP are researched and developed such as Filter Bank CSP, Regularized CSP, Invariant CSP\dots to overcome the disadvantages of vannila CSP.

\section{Feature Selection} 

\section{Classification}

\section*{Summary}