\section*{Introduction}
Human computer interaction has been evolving over the past decades, with evolving human needs. With developments in technologies, the physical 
contact between the computer and the human is decreasing rapidly. Advanced systems which work on speech and gesture control still requires a minimal
effort from the user to interact with the machine. Though these effort seem to be mere, it is a challenging task for humans with disabilities. 
Systems which work on facial gestures bridge this gap to an extent but it does not completely what is the actual intent of the person. Brain Computer
interface paves way to encode the persons intent and thoughts without the need for any physical effort. It provides enourmous capabilities for 
physicaly challenged people to express themselves just by their thoughts. 

Autonomous vehicles are the future of mobility, several companies around the world invest and research on new technologies to solve new challenges 
that appears in developing level 5 autonomy. The level of human interaction with the vehicle has been decreasing with increasing safety. However
including human in the loop is necessary at certain times to avoid any undesirable events. Level  5 autonomous vehicles is still a long way to go, 
but by bringing in a minimal interaction of the driver with the system, safety can be ensured.

\section{Brain Signal Acquisition}
    
    \subsection{Invasive Approaches}

    \subsection{NonInvase Approaches}

\section{EEG Paradigms}

    \subsection{Event Related Potential(ERP)}

    \subsection{Error Related Potential(ErRP)}

    \subsection{Motor Imagery(MI) or Oscillatory Potential}


\section*{Summary}